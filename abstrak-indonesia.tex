\begin{abstractind}
Gas LPG 3 Kg sekarang telah menjadi kebutuhan utama bagi masyarakat di Indonesia sejak pemerintah melakukan konversi dari minyak tanah ke LPG (\textit{Liquified Petroleum Gas}) 3 kilogram (Kg) dan telah mampu memberikan penghematan yang signifikan pada kas negara. Namun sekarang ketersediaan tabung gas LPG subsidi 3 Kg semakin berkurang menyebabkan pertamina selaku distributor memberlakukan kebijakan kepada Pangkalan Gas untuk wajib mengisi \textit{Logbook} Distribusi Tabung Gas 3 Kg. Dengan diberlakukannya kebijakan baru ini, ada beberapa permasalahan yang muncul seperti penerapan \textit{Logbook} ini masih juga mengalami kendala karena pelaporan gas LPG 3 Kg masih di lakukan secara manual sehingga terkadang banyak data penyaluran gas yang salah atau tidak sesuai dengan stok di pangkalan, dalam hal registrasi pembeli baru masih dilakukan dengan cara mengisi formulir yang sangat banyak sehingga proses registrasi memerlukan waktu yang lama. Solusi yang dapat dilakukan yaitu membuat sebuah aplikasi yang mampu melakukan pendataan pada penyaluran gas LPG subsidi 3 Kg ke masyarakat dan melakukan verifikasi pada data tersebut. Aplikasi ini terdiri dari dua platform yaitu platform web dibuat dengan java dan android yang dibuat menggunakan \textit{ionic framework}. Proses pembuatan aplikasi ini menggunakan metode scrum. Pengujian aplikasi menggunakan 3 metode yaitu \textit{whitebox, blackbox} dan \textit{usability testing}. Hasil pengujian \textit{blackbox} bernilai valid untuk semua fitur yang diuji. Hasil pengujian \textit{whitebox} berhasil memberikan \textit{output} yang sesuai. Pengujian \textit{usability testing} dengan 8 orang responden mendapatkan hasil 77\% untuk web dan 78\% untuk android yang masuk dalam rentang 61\%-80\% dengan nilai interpretasi skor "Layak". Berdasarkan hasil tersebut, aplikasi ini terintegrasi dengan baik dan sesuai dengan kebutuhan setiap pengguna. 


\bigskip
\noindent
\textbf{Kata kunci :} Agen Gas LPG, Gas LPG 3 Kg, Android, Ionic framework, Scrum, \textit{Blackbox}, \textit{Whitebox}, \textit{Usability testing}.
\end{abstractind}