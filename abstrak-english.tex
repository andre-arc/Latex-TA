\begin{abstracteng}
\textit{3 Kg LPG gas has now become a major requirement for people in Indonesia since the government has converted from kerosene to LPG (Liquified Petroleum Gas) 3 kilograms (Kg) and has been able to provide significant savings to the state treasury. However, now the availability of subsidized LPG gas cylinders of 3 kg is decreasing causing Pertamina as a distributor to apply a policy to the Gas Base to be obliged to fill out the Logbook for the Distribution of 3 Kg Gas Cylinders. With the enactment of this new policy, there are several problems that arise such as the application of the Logbook, which are still experiencing obstacles because reporting of 3 Kg LPG gas is still done manually, so sometimes a lot of gas distribution data is wrong or not in accordance with the stock at the base, in terms of buyer registration it is still done by filling out very many forms so that the registration process takes a long time. The solution that can be done is to make an application that is able to collect data on LPG subsidized gas distribution of 3 kg to the community and verify the data. This application consists of two platforms namely a web platform created with java and android that are created using the ionic framework. The process of making this application uses the Scrum method. Testing applications using 3 methods, namely whitebox, blackbox and usability testing. The blackbox test results are valid for all the features tested. The whitebox test results successfully provide the appropriate output. Usability testing with 8 respondents got 77\% for the web and 78\% for android in the range of 61\% -80\% with a score of "Eligible" score. Based on these results, this application is well integrated and in accordance with the needs of each user.}

\bigskip
\noindent
\textbf{\emph{Keywords :}} \textit{LPG Gas Agent, 3 Kg LPG Gas, Android, Ionic Framework, Scrum , Blackbox, Whitebox, Usability testing}
\end{abstracteng}